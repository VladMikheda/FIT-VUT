\documentclass[a4paper,twocolumn,11pt]{article}
\usepackage[left=1.5cm,text={18cm,25cm},top=2.5cm]{geometry}

\usepackage[czech]{babel}
\usepackage[utf8]{inputenc}
\usepackage[IL2]{fontenc}
\usepackage{times}


\usepackage[unicode, pdftex]{hyperref}
\usepackage{amsmath}
\usepackage{amsthm}
\usepackage{amsfonts}

\newcommand{\czuv}[1]{\quotedblbase #1\textquotedblleft}

\theoremstyle{definition}
\newtheorem{definition}{Definice}

\theoremstyle{plain}
\newtheorem{sentence}{Věta}
%----------------------------------------------------------%
\begin{document}
\begin{titlepage}
    \begin{center}
        \Huge
        \textsc{Fakulta informačních technologií\\
                vysoké učení technické v Brně}\\
        \LARGE  
        \vspace{\stretch{0.382}}
                Typografie a publikování\,--\,2. projekt\\
                Sazba dokumentů a matematických výrazů\\
        \vspace{\stretch{0.618}}
    \end{center}
    {\Large 2021 \hfill
                Mikheda Vladislav(xmikhe00)}
\end{titlepage}

\section*{Úvod}
V této úloze si vyzkoušíme sazbu titulní strany, matematic\-kých vzorců, prostředí a dalších textových struktur obvyklých pro technicky zaměřené texty (například rovnice \eqref{equation:1}
nebo Definice \ref{definition:1} na straně \pageref{definition:1}). Rovněž si vyzkoušíme používání odkazů \verb|\ref| a \verb|\pageref|.


Na titulní straně je využito sázení nadpisu podle op\-tického středu s využitím zlatého řezu. Tento postup byl
probírán na přednášce. Dále je použito odřádkování se
zadanou relativní velikostí 0.4 em a 0.3 em.


V případě, že budete potřebovat vyjádřit matematickou
konstrukci nebo symbol a nebude se Vám dařit jej nalézt
v samotném \LaTeX u, doporučuji prostudovat možnosti ba\-líku maker \AmS-\LaTeX.

\section{Matematický text}
Nejprve se podíváme na sázení matematických symbolů
~a výrazů v plynulém textu včetně sazby definic a vět s vy\-užitím balíku \texttt{amsthm}. Rovněž použijeme poznámku pod
čarou s použitím příkazu \verb|\footnote|. Někdy je vhodné
použít konstrukci \verb|\mbox{}|, která říká, že text nemá být
zalomen.


\begin{definition}
\label{definition:1}
Rozšířený zásobníkový automat \textsl{(RZA) 
je de\-finován jako sedmice tvaru $A = (Q, \Sigma, \Gamma, \delta, q_{0}, Z_0,F)$,
kde:}
\begin{itemize}
    \item[$\bullet$]{$Q$ \emph{je konečná množina} vnitřních (řídicích) stavů,}
    \item[$\bullet$]{$\Sigma$ \emph{je konečná} vstupní abeceda,}
    \item[$\bullet$]{$\Gamma$ \emph{je konečná} zásobníková abeceda,}
    \item[$\bullet$]{$\delta$ \emph{je přechodová} funkce $Q\times(\Sigma\cup\{\epsilon\})\times\Gamma^*\rightarrow2^{Q\times\Gamma^*}$,}
    \item[$\bullet$]{$q_{0}\in Q$ je počáteční stav, $Z_0\in\Gamma$je startovací symbol zásobníku a $F\subseteq Q$ \emph{je množina} koncových stavů.}
\end{itemize}


Nechť  $P = (Q, \Sigma, \Gamma, \delta, q_{0}, Z_0,F)$ je rozšířený zásob\-níkový automat. \emph{Konfigurací} nazveme trojici $(q,w,\alpha)\in Q \times \Sigma^* \times\Gamma^*$, 
kde $q$ je aktuální stav vnitřního řízení,
$w$ je dosud nezpracovaná část vstupního řetězce a 
$\alpha = Z_{i_{1}} Z_{i_{2}} \dots Z_{i_{k}}$ 
je obsah zásobníku\footnote{$Z_{i_{1}}$ je vrchol zásobníku}.
\end{definition}

\subsection{Podsekce obsahující větu a odkaz}
    \begin{definition}
    \label{definition:2}
        Řetězec $w$ nad abecedou $\Sigma$ je přijat RZA
        \emph{A jestliže $(q_0,w,Z_0)\overset{*}{\underset{A}{\vdash}} (q_F,\epsilon,\gamma)$ 
        pro nějaké $\gamma\in\Gamma^*$ a $q_F \in F$. 
        Množinu $L(A)\,=\{ w \mid w\mbox{ je přijat RZA A} \} \subseteq \Sigma^*$ nazýváme} jazyk přijímaný RZA A.
        \newpage
        Nyní si vyzkoušíme sazbu vět a důkazů opět s použitím
        balíku \texttt{amsthm}.
         \end{definition}
    \begin{sentence}
        Třída jazyků, které jsou přijímány ZA, odpovídá
        \textnormal{bezkontextovým jazykům.}
    \end{sentence}
    \begin{proof}
        V důkaze vyjdeme z Definice \ref{definition:1} a \ref{definition:2}.
    \end{proof}
    
\section{Rovnice a odkazy}
    Složitější matematické formulace sázíme mimo plynulý
    text. Lze umístit několik výrazů na jeden řádek, ale pak je
    třeba tyto vhodně oddělit, například příkazem \verb|\quad|. 
    
    
    $$\sqrt[i]{x_{i}^{3}}\quad\mbox{kde } x_i \mbox{ je } i 
    \mbox{-té sudé číslo splňující }\quad x_{i}^{x_{i}^{i^{2}}+2} \leq y_{i}^{x_{i}^{4}}$$
   
    
    V rovnici \eqref{equation:1} jsou využity tři typy závorek s různou
    explicitně definovanou velikostí.
    \begin{eqnarray}
        x & = & \left[\Bigl\{\big[a+b\big]*c\Bigr\}^{d}
        \oplus2\right]^{3/2}\label{equation:1}\\
        y & = & \lim _{x \rightarrow \infty} 
        \frac{\frac{1}{\log _{10} x}}{\sin ^{2} x+\cos ^{2} x}\nonumber
   \end{eqnarray}
    
    
    V této větě vidíme, jak vypadá implicitní vysázení limity 
    $\lim _{n\rightarrow \infty} f(n)$ v normálním odstavci textu. Podobně je to i s dalšími symboly jako $\prod _{i=1}^n 2^i$\,či $\bigcap _{A\in \mathcal{B}}$\,A. V pří\-padě vzorců 
    $\lim\limits _{n\rightarrow \infty} f(n)$ a 
    $\prod\limits _{i=1}^n 2^i$ jsme si vynutili méně úspornou sazbu příkazem \verb|\limits|.
    
    \begin{equation}
        \int_{b}^{a} g(x)~\mathrm{d}x = -\int\limits _{a}^{b}f(x)~\mathrm{d} x
    \end{equation}
\section{Matice}
    Pro sázení matic se velmi často používá prostředí array
    a závorky (\verb|\left|, \verb|\right|).

    $$
        \left(\begin{array}{ccc}
        a-b & \widehat{\xi + \omega} & \pi\\
        \vec{\mathbf{a}} & \overleftrightarrow{AC} & \hat{\beta}
        \end{array}\right)
        = 1 \Longleftrightarrow \mathcal{Q} = \mathbb{R}
    $$
    $$
        A =
        \begin{Vmatrix}
            a_{11}& a_{12} &\ldots & a_{1n}\\
            a_{21}& a_{22} &\ldots & a_{2n}\\
            \vdots& \vdots &\ddots & \vdots\\
            a_{m1}& a_{m2} &\ldots & a_{mn}
        \end{Vmatrix}
        =
        \begin{vmatrix}
             ~t & u~\\
             ~v & w~
        \end{vmatrix}
        = tw\!-\!uv
        $$
    Prostředí \texttt{array} lze úspěšně využít i jinde.
    
    $$
        \binom{n}{k} = 
        \left\{
            \begin{array}{cl}
                0 & \text{pro } k < 0 \text{ nebo } k > n\\
                \frac{n!}{k!(n-k)!} & \text{pro } 0\leq k \leq n.
             \end{array}
        \right.
    $$
\end{document}
