\documentclass[a4paper,11pt]{article}
\usepackage[left=2cm,text={17cm,24cm},top=3cm]{geometry}

\usepackage{times}
\usepackage[czech]{babel}
\usepackage[utf8]{inputenc}
% \usepackage[IL2]{fontenc}




\usepackage[unicode]{hyperref}
%  \usepackage{hyperref}
\usepackage{amsmath}
\usepackage{amsthm}
\usepackage{amsfonts}
\usepackage{lineno}
\usepackage{multirow}
\usepackage[linesnumbered,noline,boxed,czech,ruled,longend]{algorithm2e}
\usepackage{graphicx}
\usepackage{caption}
\usepackage{pdflscape}
\usepackage{graphicx}



\newcommand{\czuv}[1]{\quotedblbase #1\textquotedblleft}

%----------------------------------------------------------%
\begin{document}

\begin{titlepage}
    \begin{center}
        \Huge
        \textsc{Vysoké učení technické v~Brně\\
               \huge Fakulta informačních technologií}\\
               
        \vspace{\stretch{0.382}}
        \LARGE
                Typografie a~publikování \,--\,3.~projekt\\
                \Huge Tabulky~a obrazky\\
        \vspace{\stretch{0.618}}
    \end{center}
    {\Large \today \hfill
                Mikheda Vladislav}
\end{titlepage}

\section{Úvodní strana}
Název práce umístěte do zlatého řezu a nezapomeňte uvést dnešní datum a vaše jméno a příjmení.

\section{Tabulky}
Pro sázení tabulek můžeme použít buď prostředí \texttt{tabbing} nebo prostředí \texttt{tabular}.

\subsection{Prostředí \texttt{tabbing}}
Při použití \texttt{tabbing} vypadá tabulka následovně:

\begin{tabbing}
************** \= ****** \= ********\kill
{\bf Ovoce} \> {\bf Cena} \>  {\bf Množství}\\
Jablka \> 25,90 \> 3 kg\\
Hrušky \> 27,40 \> 2,5 kg\\
Vodní melouny\> 35,– \> 1 kus
\end{tabbing}\bigskip
Toto prostředí se dá také použít pro sázení algoritmů, ovšem vhodnější je použít prostředí \texttt{algorithm} nebo \texttt{algorithm2e} (viz sekce \ref{section:3}).

\subsection{Prostředí \texttt{tabular}}
Další možností, jak vytvořit tabulku, je použít prostředí \texttt{tabular}. Tabulky pak 
budou vypadat takto\footnote{Kdyby byl problem s \texttt{cline}, zkuste se podívat třeba sem: \href{http://www.abclinuxu.cz/tex/poradna/show/325037}{http://www.abclinuxu.cz/tex/poradna/show/325037}}:\medskip

\begin{table}[h]
    \catcode`\-=12
    \centering
    \begin{tabular}{|c|c|c|} \hline
        \multicolumn{1}{|c|}{} & \multicolumn{2}{c|}{{\bf Cena}}\\ \cline{2-3}
        {\bf Měna} & {\bf nákup} &{\bf prodej}\\ \hline
        EUR  & 25,227 & 26,943\\
        GBP  & 29,368 & 31,492\\
        USD  & 21,260 & 22,661\\ \hline
    \end{tabular}
    \caption{Tabulka kurzů k dnešnímu dni}
    \label{tabulka:1}
\end{table}\bigskip

\begin{table}[h]
    \catcode`\-=12
    \centering
    \begin{tabular}{|c|c|} \hline
            A      &    $\neg$ A\\\hline
        {\bf P}    &    N\\\hline
        {\bf O}    &    O\\\hline
        {\bf X}    &    X\\\hline
        {\bf N}    &    P\\ \hline
    \end{tabular}
    \begin{tabular}{|c|c|c|c|c|c|} \hline
        \multicolumn{2}{|c|}{\multirow{2}*{$ A \wedge  B$}} & \multicolumn{4}{c|}{$B$}\\\cline{3-6}
        \multicolumn{2}{|c|}{} & {\bf P}&{\bf O}&{\bf X}&{\bf N}\\ \hline
        \multirow{4}*{$A$}&{\bf P}&P&O&X&N\\\cline{2-6}
         &{\bf O}&O&O&N&N\\\cline{2-6}
         &{\bf X}&X&N&X&N\\\cline{2-6}
         &{\bf N}&N&N&N&N\\\hline
    \end{tabular}
    \begin{tabular}{|c|c|c|c|c|c|} \hline
        \multicolumn{2}{|c|}{\multirow{2}*{$ A \vee B$}} & \multicolumn{4}{c|}{$B$}\\\cline{3-6}
        \multicolumn{2}{|c|}{} & {\bf P}&{\bf O}&{\bf X}&{\bf N}\\ \hline
        \multirow{4}*{$A$}&{\bf P}&P&P&P&P\\\cline{2-6}
         &{\bf O}&P&O&P&O\\\cline{2-6}
         &{\bf X}&P&P&X&X\\\cline{2-6}
         &{\bf N}&P&O&X&N\\\hline
    \end{tabular}
   \begin{tabular}{|c|c|c|c|c|c|} \hline
        \multicolumn{2}{|c|}{\multirow{2}*{$A \rightarrow B$}} & \multicolumn{4}{c|}{$B$}\\\cline{3-6}
        \multicolumn{2}{|c|}{} & {\bf P}&{\bf O}&{\bf X}&{\bf N}\\ \hline
        \multirow{4}*{$A$}&{\bf P}&P&O&X&N\\\cline{2-6}
         &{\bf O}&P&O&P&O\\\cline{2-6}
         &{\bf X}&P&P&X&X\\\cline{2-6}
         &{\bf N}&P&P&P&P\\\hline
    \end{tabular}
    \caption{ Protože Kleeneho trojhodnotová logika už je
    \label{tabulka:2}
    \czuv{zastaralá}, uvádíme si zde příklad čtyřhodnotové
logiky}
\end{table}\bigskip
 \pagebreak

\section{Algoritmy}
\label{section:3}
Pokud budeme chtít vysázet algoritmus, můžeme použít prostředí \texttt{algorithm}\footnote{Pro nápovědu, jak zacházet s prostředím \texttt{algorithm}, můžeme zkusit tuhle stránku:\\
\href{http://ftp.cstug.cz/pub/tex/CTAN/macros/latex/contrib/algorithms/algorithms.pdf.}{http://ftp.cstug.cz/pub/tex/CTAN/macros/latex/contrib/algorithms/algorithms.pdf.}}~ 
nebo~ \texttt{algorithm2e}\footnote{Pro \texttt{algorithm2e} zase tuhle:
\href{http://ftp.cstug.cz/pub/tex/CTAN/macros/latex/contrib/algorithm2e/algorithm2e.pdf.}{http://ftp.cstug.cz/pub/tex/CTAN/macros/latex/contrib/algorithms/algorithms.pdf.}
}.
Příklad použití prostředí \texttt{algorithm2e} viz Algoritmus \ref{algoritmus:1}.\bigskip\bigskip

\begin{algorithm}[H]
\label{algoritmus:1}
\SetNlSty{}{}{:}
\KwIn{$(X_{t-1},u_t,z_t)$}
\KwOut{$X_t$}
\BlankLine
\Indp\Indpp
\SetNlSkip{-1em}
$\overline{X_t}= X_t = 0$\\
\For{$k = 1 \mbox{ \rm{to} } M$}
    { 
        $ x^{[k]}_{t} =  sample\_motion\_model(u_{t},x^{[k]}_{t-1})$ \\
    	$ \omega^{[k]}_{t} =  measurement\_model( z_{t},x^{[k]}_{t}, m_{t-1}) $ \\
    	$ m^{[k]}_{t} = updated\_occupancy\_grid( z_{t},x^{[k]}_{t}, m^{[k]}_{t-1})$ \\
    	$ \overline{X_{t}} = \overline{X_{t}} + \langle x^{[m]}_{x}, \omega^{[m]}_{t} \rangle$
    	
    }
\For{$k = 1 \mbox{ \rm{to} } M$}
{
    draw $i$ with probability $\approx \omega^{[i]}_t$\\
    add $\langle x^{[k]}_x,m^{[k]}_t\rangle$ to $X_t$
}
\KwRet{$X_t$}
\caption{FastSLAM}
\end{algorithm}\bigskip

\section{Obrázky}
Do našich článků můžeme samozřejmě vkládat obrázky. Pokud je obrázkem fotografie,
můžeme klidně použít bitmapový soubor. Pokud by to ale mělo být nějaké schéma nebo
něco podobného, je dobrým zvykem takovýto obrázek vytvořit vektorově.

\begin{figure}[h]
\center{\includegraphics[scale=0.4]{etiopan.eps}\reflectbox{\includegraphics[scale=0.4]{etiopan.eps}}}
\caption{Malý Etiopánek a jeho bratříček}
\label{obrazek:1}
\end{figure}
\newpage
Rozdíl mezi vektorovým\,\dots

\begin{figure}[h]
\center{\includegraphics[scale=0.4]{oniisan.eps}}
\caption{Vektorový obrázek}
\label{obrazek:2}
\end{figure}
\dots a bitmapovým obrázkem\,

\begin{figure}[h]
\center{\includegraphics[scale=0.6]{oniisan2.eps}}
\caption{ Bitmapový obrázek}
\label{obrazek:3}
\end{figure}

\bigskip
se projeví například při zvetšení


Odkazy (nejen ty) na obrázky \ref{obrazek:1}, \ref{obrazek:2} a \ref{obrazek:3}, na  
tabulky \ref{tabulka:1} a \ref{tabulka:2} a také na algoritmus \ref{algoritmus:1} jsou udělány pomocí 
křížových odkazů. Pak je ovšem potřeba zdrojový soubor přeložit dvakrát.

Vektorové obrázky lze vytvořit i přímo v \LaTeX u, například pomocí prostředí \texttt{picture}.
\newpage

\begin{landscape} 
\begin{figure}[h]
\setlength{\unitlength}{1mm}
\centering
\begin{picture}(210,110)
\put(0,0){\linethickness{1.5pt}\framebox(200,100){}}
    
    \linethickness{0,65mm}
    \put(44,14){\line(1,0){60}}
    \put(44,14){\line(-1,1){20}}
    \put(44,14){\line(-1,1){20}}
    \put(104,24){\line(1,0){40}}
    
    \put(44,14){\line(0,0){40}}
    \put(104,14){\line(0,0){40}}
    \put(144,24){\line(0,0){25}}
    \put(24,34){\line(0,0){30}}
    
    \put(164,24){\line(0,0){30}}
    \put(164,24){\line(1,0){6}}
    \put(170,24){\line(0,0){30}}
    \put(164,24){\line(-1,1){4}}
    \put(160,27.8){\line(0,0){26}}
    \put(20,66){\line(2,-1){28}}
    
    \put(110,54){\line(-1,1){30}}
    \put(80,84){\line(-1,-1){32}}
    \put(104,54){\line(-1,1){27}}
    \put(110,54){\line(-5,-1){6}}
    
    \put(80,84){\line(-5,1){13}}
    
    \put(41,59){\line(-1,3){9}}
    \put(32,86){\line(-1,-4){5}}
    \put(34,80.7){\line(-1,-5){2.7}}
    \put(31,67.5){\line(-3,-1){4}}
    \put(32,86){\line(1,0){36}}
    \put(28,69){\line(-3,-1){8}}
    \put(68,86){\line(-1,-1){27}}
    
    \put(149,49){\line(-1,0){45}}
    \put(149,49){\line(-1,1){5}}
    \put(170,54){\line(-1,0){26.5}}
    \put(170,54){\line(-1,1){28}}
    \put(82,82){\line(1,0){60}}
    
    
    \put(58,30){\line(1,0){35}}
    \put(58,30){\line(0,0){20}}
    \put(58,50){\line(1,0){35}}
    \put(93,30){\line(0,0){20}}
    
    
    \put(118,32){\line(1,0){15}}
    \put(118,32){\line(0,0){10}}
    \put(133,32){\line(0,0){10}}
    \put(118,42){\line(1,0){15}}
    
    \put(28,56){\line(2,-1){10}}
    \put(28,40){\line(3,-2){10}}
    \put(28,56){\line(0,-1){16}}
    \put(38,51){\line(0,-1){18}}
  
    \linethickness{1mm}
   \put(178,83){\circle{30}} 
\end{picture}
\caption{Vektorový obrázek moderního bydlení vhodného pro 21. století. (Bud'to vytvořte stejný obrázek, anebo nakreslete pomocí picture vás vlastní domov.)}
\end{figure}
\end{landscape}
\end{document}